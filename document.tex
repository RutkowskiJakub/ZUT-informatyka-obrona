\documentclass[12pt,a4paper]{article}
\usepackage[left=2.00cm, right=2.00cm, top=2.00cm, bottom=2.00cm]{geometry}
\usepackage[tiny]{titlesec}
\usepackage[utf8x]{inputenc}
\usepackage[polish]{babel}
\usepackage[T1]{fontenc}
\usepackage{ucs}
\usepackage{amsmath}
\usepackage{amsfonts}
\usepackage{graphicx}
\usepackage{multicol}

\title{Pytania i odpowiedzi na obronę pracy inżynierskiej}
\author{
	Zachodniopomorski Uniwersytet Technologiczny\\
	Wydział Informatyki\\
	Szczecin
}
\date{\today}

\begin{document}
	\maketitle

	\section{Co to jest algorytm - cechy i właściwości}
	Algorytm jest skończonym, uporządkowanym ciągiem jasno zdefiniowanych czynności, koniecznych do wykonania postawionego  zadania.
	Cechy algorytmów:
	\begin{itemize}
		\item \textbf{poprawność} - algorytm daje oczekiwane wyniki,
		\item \textbf{jednoznaczność} - zawsze daje te same wyniki przy takich samych danych wejściowych,
		\item \textbf{skończoność} - wykonuje się w skończonej liczbie kroków,
		\item \textbf{sprawność} - czasowa - szybkość działania i pamięciowa.
	\end{itemize}
	Właściwości algorytmów:
	\begin{itemize}
		\item \textbf{efektywność} - algorytm powinien osiągać efekt końcowy możliwie niskim kosztem,
		\item \textbf{zgodność ze specyfikacją},
		\item \textbf{właściwość stopu} - algorytm powinien zatrzymać się w skończonym czasie (po wykonaniu lub mimo niewykonania postawionego zadania).
	\end{itemize}

	\section{Porównać pojęcia program, algorytm, procedura, funkcja, agent programowy.}
	Odpowiedź

	\section{Rodzaje zabezpieczeń systemów komputerowych}
	\begin{itemize}
		\item \textbf{fizyczne} - wszelkie zabezpieczenia przed otwarciem pokrywy komputera, zamki, blokady, zabezpieczenia antykradzieżowe, kontrola dostępu do obiektów i pomieszczeń, systemy przeciwpożarowe,
		\item \textbf{techniczne} - software, oprogramowanie antywirusowe, kontrola dostępu, szyfrowanie informacji
		\item \textbf{organizacyjne} - regulaminy dla osób korzystających z systemów informatycznych, polityka bezpieczeństwa,
		\item \textbf{personalne} - sprawdzanie pracowników dopuszczonych do danych o szczególnym znaczeniu, przestrzeganie odpowiednich procedur zwalniania i zatrudniania pracowników, szkolenia.
	\end{itemize}

	\section{Urządzenia wejścia i wyjścia}
	Odpowiedź

	\section{Scharakteryzować architekturę klient-serwer oraz klient-broker-serwer.}
	\begin{itemize}
		\item  \textbf{Klient-serwer} - program klienta (aktywny) wysyła żądania do serwera, który te zapytania przetwarza i dostarcza odpowiednią usługę. Z reguły serwer (pasywny) jest jeden i może obsługiwać wiele klientów.
		\item \textbf{Klient-broker-serwer} - pomiędzy klientem a serwerem jest pośrednik (broker), który jest odpowiedzialny za odbieranie wszystkich wiadomości, ich filtrowanie, określanie kto jest odbiorcą wiadomości oraz ich przesyłanie. Odpowiada również za przechowywanie danych o sesjach oraz autoryzację klientów.
	\end{itemize}

	\section{Wymienić i omówić metody wdrażania systemów informatycznych.}
	Odpowiedź

	\section{Scharakteryzować podstawowe modele baz danych.}
	\begin{itemize}
		\item  \textbf{Hierarchiczny} - w tym modelu przechowywane dane są zorganizowane w postaci drzewa. Informacja jest zawarta w dokumentach oraz w strukturze drzewa (podobnej do drzewa folderów na dysku komputera).
		\item  \textbf{Sieciowy} - uogólniony model hierarchiczny, rekordy mogą przyjmować strukturę dowolnego grafu.
		\item  \textbf{Relacyjny} - rekordy są grupowane w relacje (tabele). Dla każdej relacji musi zostać wybrany klucz główny, jednoznacznie identyfikujący dany rekord. Klucz obcy pozwala na powiązanie relacji między sobą (skrót myślowy). Większość relacyjnych baz danych korzysta z języka zapytań SQL. W modelu relacyjnym abstrahujemy od kolejności wierszy (rekordów) i kolumn (pól w rekordzie). Wiersz reprezentuje jeden rekord informacji np. osobę. Liczba kolumn jest z góry ustalona. Z każdą kolumną jest związana jej nazwa oraz dziedzina, określająca zbiór wartości, jakie mogą wystąpić w kolumnie.
		\item  \textbf{Obiektowy} - dane przyjmują postać obiektów. W przeciwieństwie do modelu relacyjnego, rekordy i relacje między nimi przechowywane są bezpośrednio (w formie obiektów, czyli struktur zwanych klasami), bez podziału na wiersze i kolumny.
	\end{itemize}

	\section{Czym wyróżnia się rozproszonych system informatycznych od innych.}
	Odpowiedź

	\section{Porównaj metody analizy obiektowej i strukturalnej w projektowaniu systemów informatycznych.}
	\begin{itemize}
		\item W podejściu \textbf{strukturalnym} dąży się do formalnej analizy systemu. W wyniku tej analizy tworzone są hierarchiczne struktury, których elementami są procesy, dane i związki zachodzące między nimi. Cechą charakterystyczną tego podejścia jest oddzielne modelowanie danych i procesów, wykorzystujące diagramowe i macierzowe metody i techniki.
		\item Podstawową różnicą między podejściem strukturalnym a \textbf{obiektowym} jest zintegrowane, jednoczesne modelowanie danych i procesów dziedziny przedmiotowej. System w podejściu obiektowym stanowi kolekcję różnych rodzajów, wzajemnie powiązanych elementów zwanych obiektami, spełniających w nim określoną rolę. Pojęcia klasy i obiektu umożliwiły powiązanie atrybutów (danych) i operacji (usług) w elementy, które łatwo przenieść koncepcyjnie na obiekty świata rzeczywistego.
	\end{itemize}

	\section{Scharakteryzować standardowy język zapytań do baz danych.}
	Odpowiedź

	\section{Na czym polega polimorfizm metod w programowaniu obiektowym i po co się go stosuje?}
	Polega na przedefiniowaniu metod klasy nadrzędnej w klasie pochodnej. Polimorfizm pozwala traktować różnorodne dane w ten sam sposób. Przykładowo mamy wiele klas (dziedziczących z jednej klasy abstrakcyjnej) dla różnych rodzajów figur geometrycznych a dla każdej z tych figur możemy policzyć jej pole. Polimorfizm pozwala nam w każdej z tych klas zaimplementować metodę wirtualną o takiej samej nazwie np. ObliczPole() i w zależności od typu obiektu w momencie wywołania metody zostanie wykonana ta prawidłowa. Pozwala to na odseparowanie implementacji od interfejsu, co z kolei ułatwia rozszerzanie funkcjonalności.

	\section{Wymienić i scharakteryzować metody testowania oprogramowania.}
	Odpowiedź

	\section{Wymienić metody ochrony danych w systemach baz danych.}
	\begin{multicols}{2}
		\begin{itemize}
			\item Kontrola dostępu,
			\item audyt wykonywanych operacji,
			\item uwierzytelnianie,
			\item szyfrowanie danych,
			\item kontrola integralności danych,
			\item kopie zapasowe,
			\item replikacja danych,
			\item mechanizm transakcji,
			\item ochrona w warstwie aplikacji.
		\end{itemize}
	\end{multicols}

	\section{Rola sterowników w dostępie do baz danych.}
	Odpowiedź

	\section{Zarządzanie procesami w systemach operacyjnych.}
	Procesami zarządza planista (scheduler), który jest odpowiedzialny za rozpoczynanie, wznawianie i kończenie procesów oraz przełączanie kontekstu pomiędzy procesami. System operacyjny dostarcza mechanizmy umożliwiające komunikację między procesami oraz synchronizację. Planowanie procesów polega na wskazywaniu procesu, któremu ma być w danej chwili przydzielony procesor. W szczególności oznacza to decydowanie, kiedy i który proces ma przejść ze stanu gotowy do stanu aktywny. W systemie w każdej chwili może być aktywnych co najwyżej tyle procesów ile jest procesorów. W każdej chwili proces jest w jakimś stanie. Ten stan zmienia się w miarę postępu wykonania procesu. Oto możliwe stany:
	
	\begin{itemize}
		\item \textbf{Nowy} (proces właśnie utworzono) - może przejść jedynie do stanu gotowy. Dzieje się tak, gdy system załaduje program do pamięci.
		\item \textbf{Aktywny} (proces jest właśnie wykonywany przez procesor) - może przejść do jednego z trzech stanów:
		\begin{itemize}
			\item gotowy - gdy planista odbierze procesowi procesor,
			\item czekający - gdy proces rozpocznie oczekiwanie na jakieś zdarzenie, np. zleci operację wejścia-wyjścia i czeka na jej wykonanie,
			\item zakończony - gdy proces zakończy działanie.
		\end{itemize}
		\item \textbf{Czekający} - proces czeka na zajście jakiegoś zdarzenia (np. wykonanie operacji wejścia-wyjścia) -  może przejść jedynie do stanu gotowy. Dzieje się tak, gdy nastąpi oczekiwane przezeń zdarzenie, np. ukończenie operacji wejścia-wyjścia.
		\item \textbf{Gotowy} (proces czeka na przydzielenie mu procesora) - może przejść jedynie do stanu aktywny. Dzieje się tak, gdy moduł systemu operacyjnego zwany planistą przydzieli temu procesowi procesor.
		\item \textbf{Zakończony} (proces zakończył działanie) - nie może już zmienić swojego stanu.
	\end{itemize}

	\section{Co to jest system komputerowy, informacyjny, informatyczny.}
	Odpowiedź

	\section{Powody tworzenia systemów rozproszonych.}
	\begin{itemize}
		\item \textbf{Dzielenie zasobów} (ang. resource sharing) – wielu użytkowników systemu może korzystać z danego zasobu (np. drukarek, plików, usług, itp.).
		
		\item \textbf{Otwartość} (ang. openness) – podatność na rozszerzenia, możliwość rozbudowy systemu zarówno pod względem sprzętowym, jak i oprogramowania.
		
		\item \textbf{Współbieżność} (ang. concurrency) – zdolność do przetwarzania wielu zadań jednocześnie.
		
		\item \textbf{Skalowalność} (ang. scalability) – cecha systemu umożliwiająca zachowanie podobnej wydajności systemu przy zwiększaniu skali systemu (np. liczby procesów, komputerów, itp.).
		
		\item \textbf{Tolerowanie awarii} (ang. fault tolerance) – właściwość systemu umożliwiająca działania systemu mimo pojawiania się błędów i (lub) uszkodzeń (np. przez utrzymywanie nadmiarowego sprzętu).
		
		\item \textbf{Przezroczystość} (ang. transparency) – właściwość systemu powodująca postrzeganie systemu przez użytkownika jako całości, a nie poszczególnych składowych.
	\end{itemize}

	\section{Środowiska programistyczne stosowane do obliczeń inżynierskich.}
	Odpowiedź

	\section{Rodzaje systemów operacyjnych (klasyfikacja i charakterystyka).}
	\begin{itemize}
		\item Klasyfikacja ze względu na sposób przetwarzania:
		\begin{itemize}
			\item Systemy przetwarzania bezpośredniego - użytkownik wprowadza zadanie do systemu i oczekuje na wyniki. W trakcie przetwarzania jest zatem możliwa interakcja pomiędzy użytkownikiem a systemem (aplikacją). Użytkownik może być na przykład poproszony o wprowadzenie jakiś danych na terminalu, wybranie czegoś z menu itp.
			\item Systemy przetwarzania pośredniego -  zadanie jest realizowane w czasie wybranym przez system. Po przedłożeniu zadania ingerencja użytkownika jest niemożliwa. Wszystkie dane muszą być zatem dostępne w momencie przedkładania zadania, a jakikolwiek błąd programowy (np. niekompletność danych) oznacza konieczność przedłożenia i wykonania zadania ponownie.
		\end{itemize}
	
		\item Klasyfikacja ze względu na liczbę wykonywanych programów:
		\begin{itemize}
			\item  Systemy jednozadaniowe — niedopuszczalne jest rozpoczęcie wykonywania następnego zadania użytkownika przed zakończeniem poprzedniego.
			\item Systemy wielozadaniowe — dopuszczalne jest istnienie jednocześnie wielu zadań (procesów), którym zgodnie z pewną strategią przydzielany jest procesor.
		\end{itemize}
	
		\item Klasyfikacja ze względu na liczbę użytkowników:
		\begin{itemize}
			\item Systemy dla jednego użytkownika — zasoby przeznaczone są dla jednego użytkownika (np. w przypadku komputerów osobistych), nie ma mechanizmów autoryzacji, a mechanizmy ochrony informacji są ograniczone.
			\item Systemy wielodostępne — wielu użytkowników może korzystać ze zasobów systemu komputerowego, a system operacyjny gwarantuje ich ochronę przed nieupoważnioną ingerencją.
		\end{itemize}
	
		\item Inne:
		\begin{itemize}
			\item Systemy czasu rzeczywistego (ang. real-time systems) zorientowane na przetwarzanie z uwzględnieniem czasu zakończenie zadania, tzw. linii krytycznej (ang. deadline).
			\item Systemy sieciowe i rozproszone (ang. network and distributed systems) — umożliwiają zarządzanie zbiorem rozproszonych jednostek przetwarzających, czyli zbiorem jednostek (komputerów), które są zintegrowane siecią komputerową i nie współdzielą fizycznie zasobów.
			\item Systemy operacyjne komputerów naręcznych — tworzone dla rozwiązań typu PDA, czy telefonów komórkowych, podlegają istotnym ograniczeniom zasobowym.
		\end{itemize}
	\end{itemize}

	\section{Podać klasyfikację języków programowania.}
	Odpowiedź

	\section{Paradygmaty programowania obiektowego.}
	\begin{itemize}
		\item \textbf{Abstrakcja} - każdy obiekt w systemie służy jako model abstrakcyjnego "wykonawcy", który może wykonywać pracę, opisywać i zmieniać swój stan, oraz komunikować się z innymi obiektami w systemie, bez ujawniania, w jaki sposób zaimplementowano dane cechy.
		
		\item \textbf{Hermetyzacja} - oddzielenie „co” od „jak”. Enkapsulacja zapewnia, że obiekt nie może zmieniać stanu innych obiektów w nieokreślony sposób. Każdy typ obiektu dostarcza interfejsu, który określa sposób współpracy z innymi obiektami. Jedynie za pomocą określonych metod mamy możliwość zmienić stan obiektu, bezpośredni dostęp do zmiennych jest zabroniony.
		
		\item \textbf{Dziedziczenie(kompozycja)} - umożliwia stworzenie hierarchii obiektów w programie. Polega na przejęciu właściwości i funkcjonalności obiektów innej klasy i ewentualnej modyfikacji tych właściwości i funkcjonalności w taki sposób, by były one bardziej wyspecjalizowane.
		
		\item \textbf{Polimorfizm(wielopostaciowość)} - referencje i wskaźniki obiektów mogą dotyczyć obiektów różnego typu, a wywołanie metody dla referencji spowoduje zachowanie odpowiednie dla pełnego typu obiektu wywoływanego. Zazwyczaj można wyróżnić dwa rodzaje polimorfizmu: dynamiczne- wykonywane podczas działania programu, a także statyczne- na etapie kompilacji.
	\end{itemize}

	\section{Zadania systemu zarządzania bazami danych (DBMS).}
	Odpowiedź

	\section{Topologie sieci komputerowych.}
	Budowa (topologia) warunkowana jest przez zastosowanie sieci. Najprostsze komunikowanie się sieci można zrealizować przez jedynie połączenie komputerów, w innych przypadkach używa się urządzeń kierujących ruchem.
		
	\begin{itemize}
		\item \textbf{Topologia magistrali} (szyny, linii) - połączone jednym, współdzielonym medium.
		\begin{itemize}
			\item Zalety:
			\begin{itemize}
				\item Brak koncentratorów/przełączników
				\item Awaria węzła nie powoduje paraliżu sieci
			\end{itemize}
			\item Wady:
			\begin{itemize}
				\item Awaria kabla powoduje paraliż sieci
				\item Ograniczona możliwość rozbudowy
				\item Niska przepustowość
				\item Obsługuje tylko jeden kanał transmisyjny
			\end{itemize}
		\end{itemize}
	
		\item \textbf{Topologia gwiazdy} - posiada punkt centralny (switch, koncentrator) i gwiaździście połączone do niego komputery.
		\begin{itemize}
			\item Zalety:
			\begin{itemize}
				\item Bardzo łatwa rozbudowa sieci
				\item Awaria węzła nie powoduje paraliżu sieci
				\item Wysoka przepustowość
			\end{itemize}
			\item Wady:
			\begin{itemize}
				\item Ograniczenie odległości stacji roboczej od koncentratora
				\item Uszkodzenie koncentratora powoduje całkowity paraliż sieci
			\end{itemize}
		\end{itemize}
		
		\item \textbf{Topologia pierścienia} - komputery połączone są za pomocą jednego nośnika informacji w układzie zamkniętym - okablowanie nie ma żadnych zakończeń (tworzy krąg).
		\begin{itemize}
			\item Zalety:
			\begin{itemize}
				\item Niskie koszty budowy
			\end{itemize}
			\item Wady:
			\begin{itemize}
				\item Niska przepustowość
				\item Trudna do rozbudowy
				\item Ciężka lokalizacja uszkodzeń
				\item Uszkodzenie jednej stacji powoduje paraliż sieci
			\end{itemize}
		\end{itemize}
		
		\item \textbf{Rozszerzone topologie}:
		\begin{itemize}
			\item \textbf{Topologia siatki}
			\item \textbf{Topologia gwiazdy rozszerzonej} – posiada punkt centralny (podobnie do topologii gwiazdy) i punkty poboczne (jedna z częstszych topologii fizycznych Ethernetu)
			\item \textbf{Topologia podwójnego pierścienia} – poszczególne elementy są połączone pomiędzy sobą odcinkami tworząc dwa zamknięte pierścienie
			\item \textbf{Topologia siatki} – oprócz koniecznych połączeń sieć zawiera połączenia nadmiarowe; rozwiązanie często stosowane w sieciach, w których wymagana jest bezawaryjność
		\end{itemize}
	\end{itemize}

	\section{Podstawowe składniki sprzętowe w sieciach komputerowych.}
	Odpowiedź

	\section{Zastosowania mikroprocesorów.}
	\begin{itemize}
		\item elektronika przemysłowa - sterowniki PLC,
		\item elektronika powszechnego użytku - telefony komórkowe, zegarki, komputery
		\item telekomunikacja - routery, switche etc.,
		\item technika samochodowa - piloty, sterowniki świateł, lusterek, radia, kontrolery wtrysku,
		\item medycyna - ciśnieniomierze, EKG, USG, termometry, mierniki poziomu cukru we krwi,
		\item automatyka budynków - sterowniki klimatyzacji i rolet.
	\end{itemize}

	\section{Metody kompresji danych.}
	Odpowiedź

	\section{Sprzętowe środki przyspieszania obliczeń.}
	\begin{itemize}
		\item Wykorzystanie wielu rdzeni procesora.
		\item Wykorzystanie obliczeń na karcie graficznej GPGPU - (ang. general-purpose computation on graphics processing units) obliczenia ogólnego przeznaczenia na układach GPU, zwany także GPGP, rzadziej GP2. Technika, dzięki której GPU, zwykle zajmujący się tylko obliczeniami związanymi z grafiką komputerową, umożliwia wykonywanie obliczeń ogólnego przeznaczenia, tak jak CPU. Dzięki temu wiele obliczeń, głównie obliczenia równoległe, można przeprowadzić znacznie szybciej.
		\item Wykorzystanie układów FPGA, ASIC
		\begin{itemize}
			\item FPGA (ang. Field-Programmable Gate Array) - jeden z rodzajów układów PLD. Układy FPGA zawierają w sobie matrycę programowalnych bloków logicznych i konfigurowalnych połączeń między nimi. Układy FPGA ze względu na poziom skomplikowania oraz możliwości tych układów określa się jako najbardziej zaawansowane ze wszystkich rodzin PLD.
			\item ASIC (ang. Application Specific Integrated Circuit) - rodzaj układów scalonych, w których nie ma możliwości ich rekonfiguracji. Ich zaletami jest to, że wykonują funkcje szybciej i przy mniejszym użyciu zasobów niż w przypadku np. mikrokontrolerów. Układy ASIC mieszczą w sobie często kompletne funkcjonalności dla których byłoby konieczne użycie mikrokontrolera i dodatkowych układów, co umożliwia stworzenie kompletnego urządzenia w jednym chipie.
		\end{itemize}
	\end{itemize}

	\section{Klasyfikacja usług internetowych.}
	Odpowiedź

	\section{Budowa procesora (CPU).}
	Jeśli chodzi o budowę fizyczną, mikroprocesor to nic innego jak krzemowa płytka z milionem tranzystorów, które blokują lub umożliwiają przepływ prądu. Z tranzystorów budowane są bramki logiczne, a te z kolei są łączone w bardziej rozbudowane układy.
	
	\begin{itemize}
		\item \textbf{ALU} (ang. Arithmetic Logic Unit) - wykonuje podstawowe operacje arytmetyczne (dodawanie, odejmowanie, dzielenie oraz mnożenie oraz logiczne (OR, AND, XOR, NOT) oraz przesunięcia bitowe. ALU współpracuje z roboczym rejestrem zwanym akumulatorem (lub wieloma akumulatorami), który przechowuje jeden z operandów (argumentów) wykonywanej operacji oraz wyniku tej operacji.
		
		\item \textbf{CU} (ang. Control Unit) - dekoduje zawartość rejestru rozkazów i generuje odpowiednie sygnały sterujące zapewniające prawidłowy przebieg operacji zdefiniowanej kodem rozkazu.
		
		\item \textbf{Rejestry} (ang. Register) - komórki pamięci do przechowywania tymczasowych wyników obliczeń, adresów lokacji w pamięci RAM itp. Rejestry są najszybszym rodzajem pamięci.
		\begin{itemize}
			\item \textbf{Rejestr instrukcji IR} (ang. Instruction Register) - przechowuje aktualnie wykonywaną instrukcję.
			\item\textbf{ Licznik rozkazów PC} (ang. Program Counter) - przechowuje adres w pamięci, gdzie przechowywany jest kolejny rozkaz do pobrania. Rozkazy są przechowywane w postaci kodów binarnych.
			\item \textbf{Akumulator A} (ang. Accumulator) - przechowuje argument (operand) do operacji ALU lub wynik operacji.
			\item \textbf{Wskaźnik stosu SP} (ang. Stack Pointer) - wskazuje na szczyt stosu (adres ostatniej zapełnionej komórki stosu).
			\item \textbf{Rejestr flagowy} - przechowuje informacje dotyczące operacji ALU np. flaga przeniesienia lub pożyczki CF (ang. Carry Flag), flaga parzystości PF (ang. Parity Flag), flaga przepełnienia OF (ang. Overflow Flag) itp.
		\end{itemize}
		
		\item \textbf{Magistrale} (ang. Bus) - wewnętrzne szyny łączące.
		\begin{itemize}
			\item \textbf{szyna danych} (ang. data bus) - magistrala komunikacyjna wykorzystywana d przesyłania właściwych danych,
			\item \textbf{szyna adresowa} (ang. address bus) - łączy CPU z pamięcią. Określa pod jaki adres mają zostać wysłane dane szyną danych. Szerokość magistrali (liczba linii) określa maksymalną pojemność pamięci systemu (przestrzeń adresową)
			\item \textbf{szyna sterująca} (ang. control bus) - zapewnia regulację dostępu do szyny adresowej i szyny danych.
		\end{itemize}
	\end{itemize}

	\section{Technologie tworzenia stron internetowych.}
	Odpowiedź

	\section{Czym różnią się portal i wortal internetowy.}
	\textbf{Portal internetowy} jest to serwis informacyjny, na którym jest poruszanych wiele tematów z życia. Taki portal zawiera między innymi: aktualności z kraju i ze świata, prognozę pogody czy katalog wyszukiwarek.\\
	Z kolei \textbf{wortale} internetowe są to takie strony, na których zazwyczaj poruszany jest jeden temat lub pewien zakres tematyczny.

	\section{Przetwarzanie rozproszone – charakterystyka.}
	Odpowiedź

	\section{Przetwarzanie równoległe – charakterystyka.}
	Przetwarzanie danych równolegle przez jeden lub więcej komputerów. Procesy wykonywane są równocześnie. W przypadku wykonywania zadania na jednym komputerze, konieczna jest utylizacja wielu procesorów. Ze względu na skalę można wyróżnić obliczenia równoległe na poziomie: bitów, instrukcji, danych, zadań. Do prowadzenia obliczeń równoległych, oprócz sprzętu, konieczne są również odpowiednie algorytmy nazywane równoległymi. Są one trudniejsze w implementacji niż sekwencyjne, ponieważ współbieżność wprowadza dodatkowe możliwości popełnienia błędu. Powstają również dodatkowe problemy w uzyskaniu wysokiej wydajności z powodu dodatkowych nakładów na komunikację i konieczność synchronizacji obliczeń.

	\section{Grafika rastrowa a grafika wektorowa.}
	Odpowiedź

	\section{Porównanie modeli odniesienia: ISO/OSI oraz TCP/IP.}
	Model TCP/IP określany jest jako model protokołów. Każda z jego warstw wykonuje konkretne zadania, do realizacji który wykorzystywane są konkretne protokoły. Model ISO/OSI natomiast zwany modelem odniesienia, stosowany jest raczej do analizy, która pozwala lepiej zrozumieć procesy komunikacyjne zachodzące w sieci oraz stanowi wzór do projektowania rozwiązań sieciowych zarówno sprzętowych jak i programowych.\\
	Oba modele są do siebie dość podobne. Różnice jakie występują widoczne są w górnych warstwach gdzie w przypadku modelu ISO/OSI dokonano podziału, aż na 3 warstwy, a w przypadku modelu TCP/IP te same funkcje realizowane jest tylko poprzez jedną warstwę. Podobne różnice widać w dolnych warstwach, gdzie w modelu ISO/OSI mamy dwie oddzielne warstwy łącza danych i fizyczną, a w przypadku modelu TCP/IP tylko jedną, warstwę dostępu do sieci.
	
	\begin{multicols}{2}
		\begin{itemize}
			\item \textbf{ISO/OSI}:
			\begin{enumerate}
				\item Aplikacji
				\item Prezentacji
				\item Sesji
				\item Transportowa
				\item Sieciowa
				\item Łącza danych
				\item Fizyczna	
			\end{enumerate}
		
			\columnbreak
			
			\item \textbf{TCP/IP}:
			\begin{enumerate}
				\item Aplikacji (OSI 5, 6, 7)
				\item Transportowa (OSI 4)
				\item Internetowa (OSI 3)
				\item Dostępu do sieci (OSI 1, 2)
			\end{enumerate}
		
			\vfill\null
		\end{itemize}
	\end{multicols}
	

	\section{Zadania warstwy transportowej.}
	Odpowiedź

	\section{Charakterystyka warstwy fizycznej.}
	Odbiera ramki danych z warstwy 2, czyli warstwy łącza danych, i przesyła szeregowo, bit po bicie, całą ich strukturę oraz zawartość przez medium transmisyjne. Jest ona również odpowiedzialna za odbiór kolejnych bitów przychodzących strumieni danych.Określa w jaki sposób bity są przedstawiane w formie impulsów napięciowych, świetlnych czy też sygnałów radiowych. Kodowane bity mogą być reprezentowane poprzez zmiane amplitudy, częstotliwości lub fazy.\\	
	W specyfikacji warstwy fizycznej są opisane takie cechy jak napięcia elektryczne, taktowania zegarów, szybkość i maksymalne odległości transmisji. Warstwa fizyczna używa czterech procesów (adresowanie, enkapsulacja, routing, dekapsulacja).

	\section{Charakterystyka warstwy łącza danych.}
	Odpowiedź

	\section{Do czego służy protokół TCP, a do czego IP?}
	Odpowiedź

	\section{Rodzaje światłowodów - wady, zalety.}
	Odpowiedź

	\section{Scharakteryzować sieciowe systemy plików.}
	Odpowiedź

	\section{Wymień i opisz warstwy modelu OSI.}
	Odpowiedź

	\section{Podstawowe cechy standardów sieci bezprzewodowych WiFi.}
	Odpowiedź

	\section{Przedstawić budowę światłowodu.}
	Odpowiedź

	\section{Cechy charakterystyczne cyfrowych sieci ISDN.}
	Odpowiedź

	\section{Rodzaje i zastosowania macierzy dyskowych.}
	Odpowiedź

	\section{Zasada działania systemów klastrowych.}
	Odpowiedź

	\section{Zasada działania systemów ekspertowych.}
	Odpowiedź

	\section{Omów zasadę działania monitora (CRT lub LCD).}
	Odpowiedź

	\section{Wymienić i scharakteryzować rodzaje pamięci półprzewodnikowych}
	Odpowiedź

	\section{Przedstaw tablice prawdy AND, OR, XOR, zilustruj oznaczenie bramki, wymień przykładowe zastosowanie.}
	Odpowiedź

	\section{Wątki a procesy - na podstawie wybranego systemu. Wymienić wady, zalety.}
	Odpowiedź

	\section{Budowa typowego układu FPGA.}
	Odpowiedź

	\section{Podstawowe tryby adresowania systemów mikroprocesorowych}
	Odpowiedź

	\section{Hierarchia pamięci w systemie komputerowym, stronicowanie oraz koncepcja pamięci wirtualnej.}
	Odpowiedź

	\section{Omówić strukturę i funkcjonowanie systemu transmisyjnego.}
	Odpowiedź

	\section{Różnice między pamięcią statyczną i dynamiczną.}
	Odpowiedź

	\section{Problem synchronizacji przy transmisji danych i transmisja asynchroniczna}
	Odpowiedź

	\section{Uprawnienia plików na przykładzie systemu operacyjnego Unix/Linux.}
	Odpowiedź

	\section{Scharakteryzować sieciowe systemy plików.}
	Odpowiedź

	\section{Co to jest cykl życia oprogramowania i z jakich faz się składa?}
	Odpowiedź

	\section{Wymienić rodzaje diagramów w UML}
	Odpowiedź

	\section{Co oznaczają skróty ERD oraz DFD? Do czego się ich używa?}
	Odpowiedź

	\section{Przeciążanie funkcji i operatorów w języku C++.}
	Odpowiedź

	\section{Scharakteryzować instrukcje iteracyjne w przykładowym języku programowania}
	Odpowiedź

	\section{Omówić na czym polega przeciążanie funkcji i operatorów w języku C++.}
	Odpowiedź

	\section{Scharakteryzować mechanizmy dostępu do składowych klasy tworzonych statycznie i dynamicznie}
	Odpowiedź

	\section{Omów pojęcia agregacji i zawierania w diagramach UML.}
	Odpowiedź

	\section{Budowa i zasady działania wybranego urządzenia (drukarka laserowa, dysk twardy, pamięć USB, streamer, ect.)}
	Odpowiedź

	\section{Metody komunikacji człowiek-komputer.}
	Odpowiedź

	\section{Wymienić metody ekstrakcji wiedzy z danych.}
	Odpowiedź

	\section{Co to są drzewa decyzyjne i do czego służą?}
	Odpowiedź

	\section{Rekurencja i jej implementacja w językach wysokiego poziomu}
	Odpowiedź

	\section{Co to są algorytmy zachłanne – podać przykład takiego algorytmu.}
	Odpowiedź

	\section{Na czym polega haszowanie i gdzie ma ono zastosowanie?}
	Odpowiedź

	\section{Co to są problemy obliczeniowo trudne – podać przykład takiego problemu.}
	Odpowiedź

	\section{Maszynowa reprezentacja danych}
	Odpowiedź

	\section{Assembler, interpreter, kompilator – porównać i wyjaśnić pojęcia.}
	Odpowiedź

	\section{Zarządzanie pamięcią w Unix/Linux.}
	Odpowiedź

	\section{Zasady korzystanie z kluczy i pakietów kryptograficznych PGP (Pretty Good Privacy)}
	Odpowiedź

	\section{Metody reprezentacji wiedzy i wnioskowanie.}
	Odpowiedź

	\section{Zasady przetwarzanie transakcji w DBMS.}
	Odpowiedź

	\section{Narzędzia i środowiska wytwarzania oprogramowania.}
	Odpowiedź

	\section{Wzorce projektowe i programowe.}
	Odpowiedź

	\section{Metody podnoszenia niezawodności systemów wbudowanych.}
	Odpowiedź

	\section{Ryzyko i odpowiedzialność związana z systemami informatycznymi}
	Odpowiedź

	\section{Klasyfikacja systemów oprogramowania użytkowego.}
	Odpowiedź

	\section{Systemy wspomagające wytwarzanie oprogramowania – klasyfikacja, przykłady, funkcje.}
	Odpowiedź

	\section{Wymienić i scharakteryzować podstawowe techniki w grafice komputerowej.}
	Odpowiedź

	\section{Wymienić i scharakteryzować metody przetwarzania obrazów.}
	Odpowiedź

	\section{Zasady i metody tworzenia indeksów w bazach danych?}
	Odpowiedź

	\section{Rodzaje i sposób działania przerzutników.}
	Odpowiedź

	\section{Różnica pomiędzy automatem Mealy'ego a automatem Moore'a.}
	Odpowiedź

	\section{Różnica pomiędzy układami typu PLA a układami PAL.}
	Odpowiedź

	\section{Wymienić i omówić znanych światowych wynalazców w dziedzinie informatyki i telekomunikacji}
	Odpowiedź

	\section{Omówić sposoby prezentacji informacji oraz pojęcia informacji analogowej i cyfrowej, sygnału analogowego oraz cyfrowego.}
	Odpowiedź

	\section{Zdefiniować pojęcie widma sygnału oraz omówić numeryczne metody jego obliczania.}
	Odpowiedź

	\section{Omówić skalę decybelewą.}
	Odpowiedź

	\section{Co to jest szerokości pasma oraz przepływności kanału transmisyjnego.}
	Odpowiedź

	\section{Omówić zagadnienie modulacji, ze szczególnym uwzględnieniem modulacji cyfrowych}
	Odpowiedź

	\section{Wymienić znane media transmisyjne.}
	Odpowiedź

	\section{Omówić problem uwierzytelniania na przykładach: uwierzytelniania SYK, uwierzytelniania SYH, uwierzytelniania SYA oraz pojęcia hasła, karty magnetycznej, karta elektronicznej, karty identyfikacyjnej SIM oraz omówić techniki biometryczne.}
	Odpowiedź

	\section{Formaty danych liczbowych.}
	Odpowiedź

	\section{Omówić zasady wykonania operacji arytmetycznych w kodzie U2.}
	Odpowiedź

	\section{Omówić zasady wykonania operacji na liczbach zmiennopozycyjnych.}
	Odpowiedź

	\section{Różnice między pamięcią statyczną i dynamiczną.}
	Odpowiedź

	\section{Wymienić standardowe postacie wyrażeń boolowskich.}
	Odpowiedź

	\section{Omówić kombinacyjne i sekwencyjne układy logiczne.}
	Odpowiedź

	\section{Scharakteryzować poszczególne etapy procesu konwersji analogowo-cyfrowej.}
	Odpowiedź

	\section{Omówić ogólną charakterystykę filtrów w cyfrowych.}
	Odpowiedź

	\section{Opisać proces akwizycji i kodowania danych multimedialnych w kontekście zastosowania ich w systemach transmisji strumieniowej.}
	Odpowiedź

	\section{Wymienić i omówić podstawowe parametry stosowane przy definiowaniu jakości usług.}
	Odpowiedź

	\section{Wymienić i omówić podstawowe metody szeregowania pakietów.}
	Odpowiedź

	\section{Różnica między standardami JPEG i JPEG2000, rodzaje transformacji obrazu wykorzystywane w kodowaniu obrazów.}
	Odpowiedź

	\section{Scharakteryzować kod Graya jako przykład elementu wchodzącego w skład metod cyfrowej modulacji sygnału.}
	Odpowiedź

	\section{Różnica między kodami detekcyjnymi i korekcyjnymi - przykłady zastosowań.}
	Odpowiedź

\end{document}